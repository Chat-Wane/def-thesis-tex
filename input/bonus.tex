
\section*{Slides additionnels}

\begin{frame}{Structure de séquences}{Spécification}
  \begin{definition}[Spécification séquentielle d'une séquence]
  Soit une série d'opérations $H$ produisant la séquence
  $s(H) = \{p_1,\, p_2 \ldots p_k\}$ avec $p_{1..k} \in \mathcal{P}$ où
  $\mathcal{P}$ est un ensemble muni d'un ordre
  dense $(\mathcal{P},\,<_\mathcal{P})$ tel que : \\
  $\forall p\in\mathcal{P},\, p_\vdash <_\mathcal{P} p <_\mathcal{P} p_\dashv $
  \hfill et \ \
  $p_\vdash <_\mathcal{P} p_1 <_\mathcal{P} p_2 <_\mathcal{P} \ldots
  <_\mathcal{P} p_k <_\mathcal{P} p_\dashv$.
  
  \vspace{0.25cm}

  \noindent L'insertion d'un élément $e$ en position $i$ dans la séquence $s(H)$
  est définie de la façon suivante :
  \begin{equation}
    \small
    s(H \cup INSERT(i,\, e)) \rightarrow s(H) \cup 
    \begin{cases}
      \{p,\, p_\vdash <_\mathcal{P} p <_\mathcal{P} p_\dashv \} & i = 0 \wedge |s(H)| = 0\\
      \{p,\, p_\vdash <_\mathcal{P} p <_\mathcal{P} p_1 \} & i = 0 \wedge |s(H)|>0\\
      \{p,\, p_k <_\mathcal{P} p <_\mathcal{P} p_\dashv \} & i = k\\
      \{p,\, p_i <_\mathcal{P} p <_\mathcal{P} p_{i+1} \} & sinon
    \end{cases}
  \end{equation}

  \noindent La suppression de l'élément en position $i$ dans la séquence $s(H)$
  est définie de la façon suivante :
  \begin{equation}
    \small
    s(H \cup DELETE(i)) \rightarrow s(H) \setminus \{ p_i \}
  \end{equation}
\end{definition}
\end{frame}




%% WORST CASE SLIDE LSEQ

