\section{Structure de séquences répartie à large échelle}

\begin{frame}{Structure de séquences}{Réplication}
  \vspace{-1.5cm}
  La \textbf{réplication optimiste} \REF{} améliore la \textbf{disponibilité} d'un
  document et sa \textbf{réactivité} aux changements effectués.
  \vspace{0.75cm}

  Fonctionnement :
  \begin{enumerate}[(a)]
  \item Une modification produit un résultat;
  \item ce résultat est disséminé aux autres répliques;
  \item ces dernières exécutent -- ou intègrent -- le résultat reçu.
  \end{enumerate}

  \begin{textblock*}{\textwidth}(-0.65cm,0.4cm) 
    
\begin{tikzpicture}[scale=0.95]

  \newcommand\X{30pt};
  \newcommand\Y{30pt};
  
  \draw[->](0pt,   0pt)--(10*\X,   0pt);
  \draw[->](0pt, -1*\Y)--(10*\X, -1*\Y);
  \draw[->](0pt, -2*\Y)--(10*\X, -2*\Y);
  
  \draw[fill=black](0pt, 0pt) node[anchor=east]{éditeur 1 }circle(2pt);
  \draw[fill=black](0pt, -1*\Y) node[anchor=east]{éditeur 2 }circle(2pt);
  \draw[fill=black](0pt, -2*\Y) node[anchor=east]{éditeur 3 }circle(2pt);

  \draw(\X,2pt)--node[anchor=south]{[WERTY]}( \X,   -2pt);
  \draw(\X,2 -1*\Y)--node[anchor=south]{[WERTY]}(\X,-2 -1*\Y);
  \draw(\X,2 -2*\Y)--node[anchor=south]{[WERTY]}(\X,-2 -2*\Y);
  \footnotesize
  \draw(3* \X,2pt)--node[anchor=north]
  {(a) \textsc{insert}(Q, 0) \DARKBLUE{\textbf{produit} $resultat$}}(3 * \X,   -2pt);
%  \draw(3* \X,2 -2*\Y)--node[anchor=north]{\textsc{delete}(\DARKBLUE{\textbf{0}})}(3 * \X,-2 -2*\Y);
  \normalsize

  \draw(3* \X,2pt)--node[anchor=south]{[QWERTY]}(3 * \X,   -2pt);
%  \draw(2* \X,2 -1*\Y)--node[anchor=south]{[ ]}(2* \X,-2 -1*\Y)
%  \draw(3* \X,2 -2*\Y)--node[anchor=south]{[ERTY]}( 3 * \X,-2 -2*\Y);

  \footnotesize
  \draw[->, dashed] (5*\X, 0pt) -- (15+7*\X, -1*\Y);
  \draw[->, dashed] (5*\X, 0pt)
  node[anchor = south west]{(b) \DARKBLUE{\textbf{dissémine}} $resultat$} -- (7*\X, -2*\Y);


  \draw (15+ 7*\X, 2-1*\Y) -- (15+ 7*\X, -2-1*\Y)
  node [anchor=north]{(c)};
  \draw (7*\X, 2-2*\Y) -- (7*\X, -2-2*\Y)
  node [anchor=north]{(c) \DARKBLUE{\textbf{intègre}} $resultat$ };
  \normalsize

%  \draw[->, dashed] (5*\X, -2*\Y) -- (7*\X,  0*\Y)
%  node[anchor=south]{\textsc{delete}(\DARKBLUE{\textbf{0}})};
%  \normalsize
%  \draw[->, dashed] (5*\X, -2*\Y) -- (7*\X, -1*\Y);

  \draw(9*\X, 2 -0*\Y)--node[anchor=south]{[QWERTY]}(9*\X,-2 -0*\Y);
  \draw(9*\X, 2 -1*\Y)--node[anchor=south]{[QWERTY]}(9*\X,-2 -1*\Y);
  \draw(9*\X, 2 -2*\Y)--node[anchor=south]{[QWERTY]}(9*\X,-2 -2*\Y);


%%  \draw(9*\X, 2 -0*\Y)--node[anchor=south]{[QWERTY]}(9*\X,-2 -0*\Y);
%%  \draw(9*\X, 2 -1*\Y)--node[anchor=south]{[QWERTY]}(9*\X,-2 -1*\Y);
%%  \draw(9*\X, 2 -2*\Y)--node[anchor=south]{[QWERTY]}(9*\X,-2 -2*\Y);


%%  \draw[fill=white, very thick]
%%  (0*\X, 0*\Y) node{$p_1$} +(-5pt,-5pt) rectangle +(5pt,5pt);
%%  \draw[->](-5+\X, 5+2*\Y)to[out=120,in=30](0pt,5+2*\Y); %% 6 -> 7
\end{tikzpicture}
  \end{textblock*}
\end{frame}


\begin{frame}{Structure de séquences}{Cohérence des répliques}
  
  \vspace{-1.5cm}

  D'après Sun et al. \REF{}, l'édition collaborative temps réel nécessite un
  système préservant les trois propriétés : 

  \begin{itemize}
  \item Convergence;
  \item Causalité;
  \item Intention.  
  \end{itemize}

  \begin{textblock*}{\textwidth}(-0.65cm,0.4cm) 
    
\begin{tikzpicture}[scale=0.95]

  \newcommand\X{30pt};
  \newcommand\Y{30pt};
  
  \draw[->](0pt,   0pt)--(10*\X,   0pt);
  \draw[->](0pt, -1*\Y)--(10*\X, -1*\Y);
  \draw[->](0pt, -2*\Y)--(10*\X, -2*\Y);
  
  \draw[fill=black](0pt, 0pt) node[anchor=east]{réplique 1 }circle(2pt);
  \draw[fill=black](0pt, -1*\Y) node[anchor=east]{réplique 2 }circle(2pt);
  \draw[fill=black](0pt, -2*\Y) node[anchor=east]{réplique 3 }circle(2pt);

  \draw(\X,2pt)--node[anchor=south]{[WERTY]}( \X,   -2pt);
  \draw(\X,2 -1*\Y)--node[anchor=south]{[WERTY]}(\X,-2 -1*\Y);
  \draw(\X,2 -2*\Y)--node[anchor=south]{[WERTY]}(\X,-2 -2*\Y);
  \footnotesize
  \draw(3* \X,2pt)--node[anchor=north]
  {\textsc{insert}(Q, 0)}(3 * \X,   -2pt);
%  \draw(3* \X,2 -2*\Y)--node[anchor=north]{\textsc{delete}(\DARKBLUE{\textbf{0}})}(3 * \X,-2 -2*\Y);
  \normalsize

  \draw(3* \X,2pt)--node[anchor=south]{[QWERTY]}(3 * \X,   -2pt);
%  \draw(2* \X,2 -1*\Y)--node[anchor=south]{[ ]}(2* \X,-2 -1*\Y)
%  \draw(3* \X,2 -2*\Y)--node[anchor=south]{[ERTY]}( 3 * \X,-2 -2*\Y);

  \footnotesize
  \draw[->, dashed] (4*\X, 0pt) -- (4.5*\X, -1*\Y);
  \draw[->, dashed] (4*\X, 0pt) to[out=25,in=155] (7.5*\X, 0pt)
  to[out=-40,in=95] (8.2*\X, -2*\Y);
  
  \draw(5.5*\X, 2-1*\Y)node[anchor=south]{\normalsize[WERTY]}--(5.5*\X, -2-1*\Y)
  node[anchor=north]{\footnotesize\textsc{delete}(0)};

  \draw[->, dashed] (6.5*\X, -1*\Y) -- (7*\X, -0*\Y);
  \draw[->, dashed] (6.5*\X, -1*\Y) -- (7*\X, -2*\Y);

  \draw[->, dashed, color=darkblue] (7*\X, -2*\Y) to[out=-45,in=-135]
  node[anchor=north]{\DARKBLUE{\textbf{attend}}} (8.5*\X, -2*\Y);

  \normalsize

%  \draw[->, dashed] (5*\X, -2*\Y) -- (7*\X,  0*\Y)
%  node[anchor=south]{\textsc{delete}(\DARKBLUE{\textbf{0}})};
%  \normalsize
%  \draw[->, dashed] (5*\X, -2*\Y) -- (7*\X, -1*\Y);

  \draw(9*\X, 2 -0*\Y)--node[anchor=south]{[WERTY]}(9*\X,-2 -0*\Y);
  \draw(9*\X, 2 -1*\Y)--node[anchor=south]{[WERTY]}(9*\X,-2 -1*\Y);
  \draw(9*\X, 2 -2*\Y)--node[anchor=south]{[WERTY]}(9*\X,-2 -2*\Y);


%%  \draw(9*\X, 2 -0*\Y)--node[anchor=south]{[QWERTY]}(9*\X,-2 -0*\Y);
%%  \draw(9*\X, 2 -1*\Y)--node[anchor=south]{[QWERTY]}(9*\X,-2 -1*\Y);
%%  \draw(9*\X, 2 -2*\Y)--node[anchor=south]{[QWERTY]}(9*\X,-2 -2*\Y);


%%  \draw[fill=white, very thick]
%%  (0*\X, 0*\Y) node{$p_1$} +(-5pt,-5pt) rectangle +(5pt,5pt);
%%  \draw[->](-5+\X, 5+2*\Y)to[out=120,in=30](0pt,5+2*\Y); %% 6 -> 7
\end{tikzpicture}
  \end{textblock*}

\end{frame}


\begin{frame}{Structure de séquences}{Intention}
  
  L'effet observé sur le document lors de la génération d'une opération doit
  être également observé lors de son intégration malgré l'interférence
  d'opérations \textbf{concurrentes}.

  \vspace{0.5cm}

  \begin{itemize}
  \item Difficile à formaliser dans le cas général;
  \item L'opération doit respecter le plus possible sa spécification séquentielle \REF.
  \end{itemize}

  \vspace{0.5cm}
  
  Pour la séquence :
  \begin{itemize}
  \item \og insérer l'élément $e$ à la position $i$ dans la séquence \fg
  \item \og supprimer l'élément à la position $i$ dans la séquence \fg
  \end{itemize}

  \vspace{0.5cm}

  \begin{itemize}
    \only<1-1>{\item [$\rightarrow$]L'intention semble être liée à aux
      positions.}
    \only<2->{\item [$\rightarrow$]\sout{L'intention semble être
        liée aux \textbf{positions}.}}
    \uncover<2->{\item [$\rightarrow$] Une séquence
    se définit par un \textbf{ordre dense} sur ses éléments : les éléments 
    sont ordonnés
    et il est toujours possible d'insérer un élément entre deux autres éléments.}
  \end{itemize}
  
\end{frame}


\begin{frame}{Structure de séquences}{Complexités}

\begin{itemize}
  \only<1-1>{\item Complexité en \textbf{communication};}
  \only<2->{\item  \textbf{Complexité en communication};}
\item Complexité \textbf{spatiale} de la réplique;
\item Complexité \textbf{temporelle} d'une opération \textbf{générée} localement;
  \only<1-1>{\item Complexité \textbf{temporelle} de l'\textbf{intégration} d'une opération reçue.}
  \only<2->{\item \textbf{Complexité temporelle de l'intégration d'une opération reçue.}}
\end{itemize}

%% (TODO) maybe explain the reasons of this emphasis
\end{frame}


% \begin{frame}{Structure de séquences}{État de l'art : transformées opérationnelles}

%   \vspace{-1.5cm}

%   Ces approches \REF{} ont une signature identique à celle communément employée
%   pour les séquences : 
%   \begin{itemize}
%   \item \textsc{insert}($element,\,position$)
%   \item \textsc{delete}($position$)
%   \end{itemize}

%   \vspace{0.5cm}

%   Lors de la réception d'une opération, ses arguments sont ajustés afin qu'ils
%   s'appliquent à l'état courant de la réplique malgré les opérations effectuées
%   et intégrées en concurrence. 


%   \begin{textblock*}{\textwidth}(-0.65cm,0.4cm) 
%     
\begin{tikzpicture}[scale=0.95]

  \newcommand\X{30pt};
  \newcommand\Y{30pt};
  
  \draw[->](0pt,   0pt)--(10*\X,   0pt);
  \draw[->](0pt, -1*\Y)--(10*\X, -1*\Y);
  \draw[->](0pt, -2*\Y)--(10*\X, -2*\Y);
  
  \draw[fill=black](0pt, 0pt) node[anchor=east]{réplique 1 }circle(2pt);
  \draw[fill=black](0pt, -1*\Y) node[anchor=east]{réplique 2 }circle(2pt);
  \draw[fill=black](0pt, -2*\Y) node[anchor=east]{réplique 3 }circle(2pt);

  \draw(\X,2pt)--node[anchor=south]{[WERTY]}( \X,   -2pt);
  \draw(\X,2 -1*\Y)--node[anchor=south]{[WERTY]}(\X,-2 -1*\Y);
  \draw(\X,2 -2*\Y)--node[anchor=south]{[WERTY]}(\X,-2 -2*\Y);
  \footnotesize
  \draw(3* \X,2pt)--node[anchor=north]{\textsc{insert}(Q, 0)}(3 * \X,   -2pt);
  \draw(3* \X,2 -2*\Y)--node[anchor=north]{\textsc{delete}(\DARKBLUE{\textbf{0}})}(3 * \X,-2 -2*\Y);
  \normalsize

  \draw(3* \X,2pt)--node[anchor=south]{[QWERTY]}(3 * \X,   -2pt);
%  \draw(2* \X,2 -1*\Y)--node[anchor=south]{[ ]}(2* \X,-2 -1*\Y)
  \draw(3* \X,2 -2*\Y)--node[anchor=south]{[ERTY]}( 3 * \X,-2 -2*\Y);

  \draw[->, dashed] (5*\X, 0pt) -- (7*\X, -1*\Y);
  \draw[->, dashed] (5*\X, 0pt) -- (7*\X, -2*\Y);

  \footnotesize
  \draw[->, dashed] (5*\X, -2*\Y) -- (7*\X,  0*\Y)
  node[anchor=south]{\textsc{delete}(\DARKBLUE{\textbf{1}})};
  \normalsize
  \draw[->, dashed] (5*\X, -2*\Y) -- (7*\X, -1*\Y);

  \draw(9*\X, 2 -0*\Y)--node[anchor=south]{[QERTY]}(9*\X,-2 -0*\Y);
  \draw(9*\X, 2 -1*\Y)--node[anchor=south]{[QERTY]}(9*\X,-2 -1*\Y);
  \draw(9*\X, 2 -2*\Y)--node[anchor=south]{[QERTY]}(9*\X,-2 -2*\Y);


%%  \draw(9*\X, 2 -0*\Y)--node[anchor=south]{[QWERTY]}(9*\X,-2 -0*\Y);
%%  \draw(9*\X, 2 -1*\Y)--node[anchor=south]{[QWERTY]}(9*\X,-2 -1*\Y);
%%  \draw(9*\X, 2 -2*\Y)--node[anchor=south]{[QWERTY]}(9*\X,-2 -2*\Y);


%%  \draw[fill=white, very thick]
%%  (0*\X, 0*\Y) node{$p_1$} +(-5pt,-5pt) rectangle +(5pt,5pt);
%%  \draw[->](-5+\X, 5+2*\Y)to[out=120,in=30](0pt,5+2*\Y); %% 6 -> 7
\end{tikzpicture}
%   \end{textblock*}
% \end{frame}

\begin{frame}{Structure de séquences}{Définition du problème}

\end{frame}

