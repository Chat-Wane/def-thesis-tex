\section{Structure de séquences répartie à large échelle}

\begin{frame}{Structure de séquences}{Réplication}
  \vspace{-1.5cm}
  La \textbf{réplication optimiste} \REF{} améliore la \textbf{disponibilité} d'un
  document et sa \textbf{réactivité} aux changements effectués.
  \vspace{0.75cm}

  Fonctionnement :
  \begin{enumerate}[(a)]
  \item Une modification produit un résultat;
  \item ce résultat est disséminé aux autres répliques;
  \item ces dernières exécutent -- ou intègrent -- le résultat reçu.
  \end{enumerate}

  \begin{textblock*}{\textwidth}(-0.65cm,0.4cm) 
    
\begin{tikzpicture}[scale=0.95]

  \newcommand\X{30pt};
  \newcommand\Y{30pt};
  
  \draw[->](0pt,   0pt)--(10*\X,   0pt);
  \draw[->](0pt, -1*\Y)--(10*\X, -1*\Y);
  \draw[->](0pt, -2*\Y)--(10*\X, -2*\Y);
  
  \draw[fill=black](0pt, 0pt) node[anchor=east]{éditeur 1 }circle(2pt);
  \draw[fill=black](0pt, -1*\Y) node[anchor=east]{éditeur 2 }circle(2pt);
  \draw[fill=black](0pt, -2*\Y) node[anchor=east]{éditeur 3 }circle(2pt);

  \draw(\X,2pt)--node[anchor=south]{[WERTY]}( \X,   -2pt);
  \draw(\X,2 -1*\Y)--node[anchor=south]{[WERTY]}(\X,-2 -1*\Y);
  \draw(\X,2 -2*\Y)--node[anchor=south]{[WERTY]}(\X,-2 -2*\Y);
  \footnotesize
  \draw(3* \X,2pt)--node[anchor=north]
  {(a) \textsc{insert}(Q, 0) \DARKBLUE{\textbf{produit} $resultat$}}(3 * \X,   -2pt);
%  \draw(3* \X,2 -2*\Y)--node[anchor=north]{\textsc{delete}(\DARKBLUE{\textbf{0}})}(3 * \X,-2 -2*\Y);
  \normalsize

  \draw(3* \X,2pt)--node[anchor=south]{[QWERTY]}(3 * \X,   -2pt);
%  \draw(2* \X,2 -1*\Y)--node[anchor=south]{[ ]}(2* \X,-2 -1*\Y)
%  \draw(3* \X,2 -2*\Y)--node[anchor=south]{[ERTY]}( 3 * \X,-2 -2*\Y);

  \footnotesize
  \draw[->, dashed] (5*\X, 0pt) -- (15+7*\X, -1*\Y);
  \draw[->, dashed] (5*\X, 0pt)
  node[anchor = south west]{(b) \DARKBLUE{\textbf{dissémine}} $resultat$} -- (7*\X, -2*\Y);


  \draw (15+ 7*\X, 2-1*\Y) -- (15+ 7*\X, -2-1*\Y)
  node [anchor=north]{(c)};
  \draw (7*\X, 2-2*\Y) -- (7*\X, -2-2*\Y)
  node [anchor=north]{(c) \DARKBLUE{\textbf{intègre}} $resultat$ };
  \normalsize

%  \draw[->, dashed] (5*\X, -2*\Y) -- (7*\X,  0*\Y)
%  node[anchor=south]{\textsc{delete}(\DARKBLUE{\textbf{0}})};
%  \normalsize
%  \draw[->, dashed] (5*\X, -2*\Y) -- (7*\X, -1*\Y);

  \draw(9*\X, 2 -0*\Y)--node[anchor=south]{[QWERTY]}(9*\X,-2 -0*\Y);
  \draw(9*\X, 2 -1*\Y)--node[anchor=south]{[QWERTY]}(9*\X,-2 -1*\Y);
  \draw(9*\X, 2 -2*\Y)--node[anchor=south]{[QWERTY]}(9*\X,-2 -2*\Y);


%%  \draw(9*\X, 2 -0*\Y)--node[anchor=south]{[QWERTY]}(9*\X,-2 -0*\Y);
%%  \draw(9*\X, 2 -1*\Y)--node[anchor=south]{[QWERTY]}(9*\X,-2 -1*\Y);
%%  \draw(9*\X, 2 -2*\Y)--node[anchor=south]{[QWERTY]}(9*\X,-2 -2*\Y);


%%  \draw[fill=white, very thick]
%%  (0*\X, 0*\Y) node{$p_1$} +(-5pt,-5pt) rectangle +(5pt,5pt);
%%  \draw[->](-5+\X, 5+2*\Y)to[out=120,in=30](0pt,5+2*\Y); %% 6 -> 7
\end{tikzpicture}
  \end{textblock*}
\end{frame}


\begin{frame}{Structure de séquences}{Cohérence des répliques}
  
  \vspace{-1.5cm}

  D'après Sun et al. \REF{}, l'édition collaborative temps réel nécessite un
  système préservant les trois propriétés : 

  \begin{itemize}
  \item Convergence;
  \item Causalité;
  \item Intention.  
  \end{itemize}

  \begin{textblock*}{\textwidth}(-0.65cm,0.4cm) 
    
\begin{tikzpicture}[scale=0.95]

  \newcommand\X{30pt};
  \newcommand\Y{30pt};
  
  \draw[->](0pt,   0pt)--(10*\X,   0pt);
  \draw[->](0pt, -1*\Y)--(10*\X, -1*\Y);
  \draw[->](0pt, -2*\Y)--(10*\X, -2*\Y);
  
  \draw[fill=black](0pt, 0pt) node[anchor=east]{réplique 1 }circle(2pt);
  \draw[fill=black](0pt, -1*\Y) node[anchor=east]{réplique 2 }circle(2pt);
  \draw[fill=black](0pt, -2*\Y) node[anchor=east]{réplique 3 }circle(2pt);

  \draw(\X,2pt)--node[anchor=south]{[WERTY]}( \X,   -2pt);
  \draw(\X,2 -1*\Y)--node[anchor=south]{[WERTY]}(\X,-2 -1*\Y);
  \draw(\X,2 -2*\Y)--node[anchor=south]{[WERTY]}(\X,-2 -2*\Y);
  \footnotesize
  \draw(3* \X,2pt)--node[anchor=north]
  {\textsc{insert}(Q, 0)}(3 * \X,   -2pt);
%  \draw(3* \X,2 -2*\Y)--node[anchor=north]{\textsc{delete}(\DARKBLUE{\textbf{0}})}(3 * \X,-2 -2*\Y);
  \normalsize

  \draw(3* \X,2pt)--node[anchor=south]{[QWERTY]}(3 * \X,   -2pt);
%  \draw(2* \X,2 -1*\Y)--node[anchor=south]{[ ]}(2* \X,-2 -1*\Y)
%  \draw(3* \X,2 -2*\Y)--node[anchor=south]{[ERTY]}( 3 * \X,-2 -2*\Y);

  \footnotesize
  \draw[->, dashed] (4*\X, 0pt) -- (4.5*\X, -1*\Y);
  \draw[->, dashed] (4*\X, 0pt) to[out=25,in=155] (7.5*\X, 0pt)
  to[out=-40,in=95] (8.2*\X, -2*\Y);
  
  \draw(5.5*\X, 2-1*\Y)node[anchor=south]{\normalsize[WERTY]}--(5.5*\X, -2-1*\Y)
  node[anchor=north]{\footnotesize\textsc{delete}(0)};

  \draw[->, dashed] (6.5*\X, -1*\Y) -- (7*\X, -0*\Y);
  \draw[->, dashed] (6.5*\X, -1*\Y) -- (7*\X, -2*\Y);

  \draw[->, dashed, color=darkblue] (7*\X, -2*\Y) to[out=-45,in=-135]
  node[anchor=north]{\DARKBLUE{\textbf{attend}}} (8.5*\X, -2*\Y);

  \normalsize

%  \draw[->, dashed] (5*\X, -2*\Y) -- (7*\X,  0*\Y)
%  node[anchor=south]{\textsc{delete}(\DARKBLUE{\textbf{0}})};
%  \normalsize
%  \draw[->, dashed] (5*\X, -2*\Y) -- (7*\X, -1*\Y);

  \draw(9*\X, 2 -0*\Y)--node[anchor=south]{[WERTY]}(9*\X,-2 -0*\Y);
  \draw(9*\X, 2 -1*\Y)--node[anchor=south]{[WERTY]}(9*\X,-2 -1*\Y);
  \draw(9*\X, 2 -2*\Y)--node[anchor=south]{[WERTY]}(9*\X,-2 -2*\Y);


%%  \draw(9*\X, 2 -0*\Y)--node[anchor=south]{[QWERTY]}(9*\X,-2 -0*\Y);
%%  \draw(9*\X, 2 -1*\Y)--node[anchor=south]{[QWERTY]}(9*\X,-2 -1*\Y);
%%  \draw(9*\X, 2 -2*\Y)--node[anchor=south]{[QWERTY]}(9*\X,-2 -2*\Y);


%%  \draw[fill=white, very thick]
%%  (0*\X, 0*\Y) node{$p_1$} +(-5pt,-5pt) rectangle +(5pt,5pt);
%%  \draw[->](-5+\X, 5+2*\Y)to[out=120,in=30](0pt,5+2*\Y); %% 6 -> 7
\end{tikzpicture}
  \end{textblock*}

\end{frame}


\begin{frame}{Structure de séquences}{Intention}
  
  L'effet observé sur le document lors de la génération d'une opération doit
  être également observé lors de son intégration malgré l'interférence
  d'opérations \textbf{concurrentes}.

  \vspace{0.5cm}

  \begin{itemize}
  \item Difficile à formaliser dans le cas général;
  \item L'opération doit respecter le plus possible sa spécification séquentielle \REF.
  \end{itemize}

  \vspace{0.5cm}
  
  Pour la séquence :
  \begin{itemize}
  \item \og insérer l'élément $e$ à la position $i$ dans la séquence \fg
  \item \og supprimer l'élément à la position $i$ dans la séquence \fg
  \end{itemize}

  \vspace{0.5cm}

  \begin{itemize}
    \only<1-1>{\item [$\rightarrow$]L'intention semble être liée à aux
      positions.}
    \only<2->{\item [$\rightarrow$]\sout{L'intention semble être
        liée aux \textbf{positions}.}}
    \uncover<2->{\item [$\rightarrow$] Une séquence
    se définit par un \textbf{ordre dense} sur ses éléments : les éléments 
    sont ordonnés
    et il est toujours possible d'insérer un élément entre deux autres éléments.}
  \end{itemize}
  
\end{frame}


\begin{frame}{Structure de séquences}{Complexités}

\begin{itemize}
  \only<1-1>{\item Complexité en \textbf{communication};}
  \only<2->{\item  \textbf{Complexité en communication};}
\item Complexité \textbf{spatiale} de la réplique;
\item Complexité \textbf{temporelle} d'une opération \textbf{générée} localement;
  \only<1-1>{\item Complexité \textbf{temporelle} de l'\textbf{intégration} d'une opération reçue.}
  \only<2->{\item \textbf{Complexité temporelle de l'intégration d'une opération reçue.}}
\end{itemize}

%% (TODO) maybe explain the reasons of this emphasis
\end{frame}


% \begin{frame}{Structure de séquences}{État de l'art : transformées opérationnelles}

%   \vspace{-1.5cm}

%   Ces approches \REF{} ont une signature identique à celle communément employée
%   pour les séquences : 
%   \begin{itemize}
%   \item \textsc{insert}($element,\,position$)
%   \item \textsc{delete}($position$)
%   \end{itemize}

%   \vspace{0.5cm}

%   Lors de la réception d'une opération, ses arguments sont ajustés afin qu'ils
%   s'appliquent à l'état courant de la réplique malgré les opérations effectuées
%   et intégrées en concurrence. 


%   \begin{textblock*}{\textwidth}(-0.65cm,0.4cm) 
%     
\begin{tikzpicture}[scale=0.95]

  \newcommand\X{30pt};
  \newcommand\Y{30pt};
  
  \draw[->](0pt,   0pt)--(10*\X,   0pt);
  \draw[->](0pt, -1*\Y)--(10*\X, -1*\Y);
  \draw[->](0pt, -2*\Y)--(10*\X, -2*\Y);
  
  \draw[fill=black](0pt, 0pt) node[anchor=east]{réplique 1 }circle(2pt);
  \draw[fill=black](0pt, -1*\Y) node[anchor=east]{réplique 2 }circle(2pt);
  \draw[fill=black](0pt, -2*\Y) node[anchor=east]{réplique 3 }circle(2pt);

  \draw(\X,2pt)--node[anchor=south]{[WERTY]}( \X,   -2pt);
  \draw(\X,2 -1*\Y)--node[anchor=south]{[WERTY]}(\X,-2 -1*\Y);
  \draw(\X,2 -2*\Y)--node[anchor=south]{[WERTY]}(\X,-2 -2*\Y);
  \footnotesize
  \draw(3* \X,2pt)--node[anchor=north]{\textsc{insert}(Q, 0)}(3 * \X,   -2pt);
  \draw(3* \X,2 -2*\Y)--node[anchor=north]{\textsc{delete}(\DARKBLUE{\textbf{0}})}(3 * \X,-2 -2*\Y);
  \normalsize

  \draw(3* \X,2pt)--node[anchor=south]{[QWERTY]}(3 * \X,   -2pt);
%  \draw(2* \X,2 -1*\Y)--node[anchor=south]{[ ]}(2* \X,-2 -1*\Y)
  \draw(3* \X,2 -2*\Y)--node[anchor=south]{[ERTY]}( 3 * \X,-2 -2*\Y);

  \draw[->, dashed] (5*\X, 0pt) -- (7*\X, -1*\Y);
  \draw[->, dashed] (5*\X, 0pt) -- (7*\X, -2*\Y);

  \footnotesize
  \draw[->, dashed] (5*\X, -2*\Y) -- (7*\X,  0*\Y)
  node[anchor=south]{\textsc{delete}(\DARKBLUE{\textbf{1}})};
  \normalsize
  \draw[->, dashed] (5*\X, -2*\Y) -- (7*\X, -1*\Y);

  \draw(9*\X, 2 -0*\Y)--node[anchor=south]{[QERTY]}(9*\X,-2 -0*\Y);
  \draw(9*\X, 2 -1*\Y)--node[anchor=south]{[QERTY]}(9*\X,-2 -1*\Y);
  \draw(9*\X, 2 -2*\Y)--node[anchor=south]{[QERTY]}(9*\X,-2 -2*\Y);


%%  \draw(9*\X, 2 -0*\Y)--node[anchor=south]{[QWERTY]}(9*\X,-2 -0*\Y);
%%  \draw(9*\X, 2 -1*\Y)--node[anchor=south]{[QWERTY]}(9*\X,-2 -1*\Y);
%%  \draw(9*\X, 2 -2*\Y)--node[anchor=south]{[QWERTY]}(9*\X,-2 -2*\Y);


%%  \draw[fill=white, very thick]
%%  (0*\X, 0*\Y) node{$p_1$} +(-5pt,-5pt) rectangle +(5pt,5pt);
%%  \draw[->](-5+\X, 5+2*\Y)to[out=120,in=30](0pt,5+2*\Y); %% 6 -> 7
\end{tikzpicture}
%   \end{textblock*}
% \end{frame}


\begin{frame}{Structure de séquences}{Le prix de la causalité}
  
  Ordonner les opérations selon l'ordre causal est extrêmement coûteux : au
  minimum $\mathcal{O}(W)$ en communication où $W$ est le nombre de participants
  ayant jamais écrit dans le document \REF.
  \begin{itemize}
  \item[$\rightarrow$] Relaxer l'ordre est nécessaire pour le passage à l'échelle
  \end{itemize}
  
  \vspace{0.5cm}
  
  \large
  \begin{itemize}
  \item [$\rightarrow$] L'utilisation de structures de données répliquées sans
    conflits \REF permet cela.
    \begin{itemize}
    \item [$\rightarrow$] Ces approches se basent sur la génération
      d'identifiants uniques et immuables. Nous nous intéresserons
      particulièrement à celles dont les identifiants sont des listes de taille
      variable.
    \end{itemize}
  \end{itemize}

\end{frame}

\begin{frame}{Structure de séquences}{Structures sans conflits}
  
  La signature change : 
  \begin{itemize}
  \item \textsc{insert}($element,\, position$) $\rightarrow$
    \textsc{insert}($id_{position-1},\, element,\, id_{position}$)
  \item \textsc{delete}($position$) $\rightarrow$ \textsc{delete}($id_{position}$)
  \end{itemize}

  \vspace{0.5cm}
  

  \begin{algorithm}[H]
    
\footnotesize
\algrenewcommand{\algorithmiccomment}[1]{\hskip2em$\rhd$ #1}

\newcommand{\comment}[1]{$\rhd$ #1}


\algblockdefx[initially]{initially}{endInitially}
  [0] {\textbf{INITIALLY:}} 

\algblockdefx[local]{local}{endLocal}
  [0] {\textbf{LOCAL UPDATE:}}

\algsetblockdefx[received]{received}{endReceived}
  {65535}{}
  [0] {\textbf{RECEIVED UPDATE:}}

\algblockdefx[onInsert]{onLocal}{endOnLocal}
  [0] {\textbf{on} insert ($previous \in \mathcal{I},\,\alpha \in \mathcal{A},\,
   next\in\mathcal{I}$):}
  [0] {\textbf{on} delete ($i \in \mathcal{I}$):} 

\algsetblockdefx[onRemote]{onRemote}{endOnRemote}
  {65535}{}
  [0] {\textbf{on} insert ($i\in\mathcal{I}$):
    \hfill\comment{\DARKBLUE{\textbf{une fois}} par identifiant}}
  [0] {\textbf{on} delete ($i\in\mathcal{I}$):
    \hfill\comment{\DARKBLUE{\textbf{après}} l'exécution de \textsc{insert}($i$)}} 

\newcommand{\LINEFOR}[2]{%
  \algorithmicfor\ {#1}\ \algorithmicdo\ {#2} %
  }

\newcommand{\LINEIFTHEN}[2]{%
  \algorithmicif\ {#1}\ \algorithmicthen\ {#2} %
  }

\newcommand{\INDSTATE}[1][1]{\State\hspace{\algorithmicindent}}

\begin{algorithmic}[1]
  \Statex
  \initially
    \State $T \leftarrow \varnothing$;
    \hfill \comment{CRDT conçue pour les séquences}
  \endInitially
  
  \local
    \onLocal
    \State \textbf{let}
    $\langle p,\, q \rangle \leftarrow $\textsc{convert2Path}$(previous,\, next)$;
    \State \textbf{let}
    $\DARKBLUE{newPath \leftarrow} $\DARKBLUE{\textbf{\textsc{allocPath}}}\DARKBLUE{$(p,\,q)$}; \label{line:allocpath}
    \State \textbf{let} 
    $newDis \leftarrow $\textsc{allocDis}$(p,\, newPath,\, q)$; \label{line:allocdes}
    \State \textsc{broadcast}$(\text{'insert'},\,
    \langle newPath,\, \alpha,\, newDis \rangle)$;
    \endOnLocal
    \INDSTATE \textsc{broadcast}$(\text{'delete'},\,i)$;
  \endLocal
  
  \received
    \onRemote
    \State $T \leftarrow T \cup i$;
    \endOnRemote
    \INDSTATE $T \leftarrow T\, \backslash\, i$; 
  
\end{algorithmic}

  \end{algorithm}

\end{frame}


\begin{frame}{Structure de séquences}{Problèmes d'allocation}

  La fonction d'allocation ne connait ni le \textbf{nombre} de caractères, ni
  leur \textbf{position} dans la séquence finale.

  \vspace{0.5cm}

  \begin{center}
    \begin{tikzpicture}

  \draw(-4,1)node{0};

  \draw[hide on=1-5, bold on=6](-3,1)node{0.0.0.1};
  \draw[hide on=1-4, bold on=5](-2,1)node{0.0.1};
  \draw[hide on=1-3, bold on=4](-1,1)node{0.1};
  \draw[bold on=1](0,1)node{1};
  \draw[hide on=1, bold on=2](1,1)node{2};
  \draw[hide on=1-2,bold on=3](2,1)node{3};
%  \draw[hide on=1-2,bold on=3](3,1)node{8};
%  \draw[hide on=1-3,bold on=4](4,1)node{8.1};

  \draw(5,1)node{9};

  \draw[hide on=1-5, bold on=6](-3,0)node{Q};
  \draw[hide on=1-4, bold on=5](-2,0)node{W};
  \draw[hide on=1-3, bold on=4](-1,0)node{E};
  \draw[bold on=1](0,0)node{R};
  \draw[hide on=1, bold on=2](1,0)node{T};
  \draw[hide on=1-2, bold on=3](2,0)node{Y};
%  \draw[hide on=1-2, bold on=3](3,0)node{E};
%  \draw[hide on=1-2, bold on=3](4,0)node{S};


\end{tikzpicture}
  \end{center}

\end{frame}


\begin{frame}{Structure de séquences}{Arbre}
  
  \begin{center}
    \begin{tikzpicture}[scale=1.2]

\newcommand\Y{-30}
\newcommand\ADDY{-8}


  %% node to node
  \small
  \draw[thick] (0pt,0pt) -- node[anchor=south east]{\DARKBLUE{0}} (-40pt,\Y pt);
  \draw[thick] (0pt,0pt) -- node[anchor=east]{3} (30pt, \Y pt); %% Y
  \draw[thick] (0pt, 0 pt) -- node[anchor=east]{2} (15pt, 1 * \Y pt); %% T
  \draw[thick] (-40pt, \Y pt) -- node[anchor=east]{\DARKBLUE{0}} (-40pt, 2 * \Y pt); %% 0
  \draw[thick] (0pt, 0 pt) -- node[anchor=east]{1} (0pt, 1 * \Y pt); %% R
  \draw[thick] (-40pt, 2*\Y pt)-- node[anchor=east]{\DARKBLUE{0}}(-40pt, 3 * \Y pt); %% 0
  \draw[thick] (-40pt, 1*\Y pt) -- node[anchor=north]{1}(-15pt, 2*\Y pt); %% E
  % \draw[thick] (-40pt, 3*\Y pt) -- node[anchor=east]{\DARKBLUE{0}}(-40pt,4 * \Y pt); %% 0
  \draw[thick] (-40pt, 2*\Y pt) -- node[anchor=north]{1}(-25pt, 3* \Y pt); %% W
  % \draw[thick] (-40pt, 4*\Y pt) -- node[anchor=east]{\DARKBLUE{0}}(-40pt,5 * \Y pt); %% 0
  \draw[thick] (-40pt, 3*\Y pt) -- node[anchor=east]{\DARKBLUE{1}}(-35pt, 4*\Y pt); %% Q

  \draw[dashed, thick] (0pt,0pt) -- node[anchor=south west]{9} (40pt,\Y pt);

  %% node to element
  \draw[->] ( 30pt, \Y pt) -- ( 30pt, \ADDY + \Y pt); %% Y
  \draw[->] ( 15pt, 1* \Y pt) -- ( 15pt, \ADDY +  1*\Y pt); %% T
  \draw[->] (  0pt, 1 *\Y pt) -- (  0pt, \ADDY +  1 *\Y pt); %% R
  \draw[->] (-15pt, 2 *\Y pt) -- ( -15pt, \ADDY + 2 *\Y pt); %% E
  \draw[->] (-25pt, 3 *\Y pt) -- ( -25pt, \ADDY + 3 *\Y pt); %% W
  \draw[->] (-35pt, 4 *\Y pt) -- ( -35pt, \ADDY + 4 *\Y pt); %% Q

  %% element to desambiguator
  % \draw[->,densely dashdotted]
  % ( 30pt, \ADDY + \Y pt) -- ( 30pt,2.75*\ADDY+\Y pt); %% Y
  % \draw[->,densely dashdotted]
  % ( 15pt, \ADDY + 2* \Y pt) -- ( 15pt,2.75*\ADDY+ 2* \Y pt); %% T
  % \draw[->,densely dashdotted]
  % ( 0pt, \ADDY + 3* \Y pt) -- (  0pt,2.75*\ADDY+ 3* \Y pt); %% R
  % \draw[->,densely dashdotted]
  % ( -15pt, \ADDY + 4 *\Y pt) -- ( -15pt,2.75*\ADDY+ 4* \Y pt); %% E
  % \draw[->,densely dashdotted]
  % ( -25pt, \ADDY + 5 *\Y pt) -- ( -25pt,2.75*\ADDY+ 5*\Y pt); %% W
  % \draw[->,densely dashdotted]
  % ( -35pt, \ADDY + 6* \Y pt) -- ( -35pt,2.75*\ADDY+ 6*\Y pt); %% Q

  %% node
 \draw[fill=black] (0pt,0pt) circle (1pt); %% rooot
  \draw[fill=white] ( 30pt, \Y pt) circle (1pt); %% Y
  \draw[fill=white] (-40pt, \Y pt) circle (1pt); %% 0
  \draw[fill=white] ( 15 pt, 1 * \Y pt) circle (1pt); %% T
  \draw[fill=white] (-40pt, 2 * \Y pt) circle (1pt); %% 0
  \draw[fill=white] (  0 pt, 1 * \Y pt) circle (1pt); %% R
  \draw[fill=white] (-40pt, 3 * \Y pt) circle (1pt); %% 0
  \draw[fill=white] (-15 pt, 2 * \Y pt) circle (1pt); %% E
%  \draw[fill=white] (-40pt, 4 * \Y pt) circle (1pt); %% 0
  \draw[fill=white] (-25 pt, 3 * \Y pt) circle (1pt); %% W
%  \draw[fill=white] (-40pt, 5 * \Y pt) circle (1pt); %% 0
  \draw[fill=white] (-35 pt, 4 * \Y pt) circle (1pt); %% Q

  \draw[fill=black] ( 40pt, \Y pt) circle (1pt);


  %% elements
  \draw[fill=white] ( 30pt, -4 + \ADDY + \Y pt)
  node{\textbf{Y}} +(-4pt,-4pt) rectangle +(4pt,4pt) ; %% Y
  \draw[fill=white] ( 15pt, -4 + \ADDY +  1 *\Y pt)
  node{\textbf{T}} +(-4pt,-4pt) rectangle +(4pt,4pt) ; %% T
  \draw[fill=white] (  0pt, -4 + \ADDY +  1* \Y pt)
  node{\textbf{R}} +(-4pt,-4pt) rectangle +(4pt,4pt) ; %% R
  \draw[fill=white] (-15pt, -4 + \ADDY + 2 *\Y pt)
  node{\textbf{E}} +(-4pt,-4pt) rectangle +(4pt,4pt) ; %% E
  \draw[fill=white] (-25pt, -4 + \ADDY + 3 * \Y pt)
  node{\textbf{W}} +(-4pt,-4pt) rectangle +(4pt,4pt) ; %% W
  \draw[fill=white] (-35pt, -4 + \ADDY + 4 *\Y pt)
  node{\textbf{Q}} +(-4pt,-4pt) rectangle +(4pt,4pt) ; %% Q

  %% desambiguator
  % \draw[fill=gray!20]( 30pt, -2.5 + 2.75 * \ADDY + \Y pt)
  % +(-2.5pt,-2.5pt) rectangle +(2.5pt,2.5pt);
  % \draw[fill=gray!20]( 15pt, -2.5 + 2.75 * \ADDY +2 *\Y pt)
  % +(-2.5pt,-2.5pt) rectangle +(2.5pt,2.5pt);
  % \draw[fill=gray!20](  0pt, -2.5 + 2.75 * \ADDY + 3*\Y pt)
  % +(-2.5pt,-2.5pt) rectangle +(2.5pt,2.5pt);
  % \draw[fill=gray!20](-15pt, -2.5 + 2.75 * \ADDY +4*\Y pt )
  % +(-2.5pt,-2.5pt) rectangle +(2.5pt,2.5pt);
  % \draw[fill=gray!20](-25pt, -2.5 + 2.75 * \ADDY + 5*\Y pt)
  % +(-2.5pt,-2.5pt) rectangle +(2.5pt,2.5pt);
  % \draw[fill=gray!20](-35pt, -2.5 + 2.75 * \ADDY +6*\Y pt) 
  % +(-2.5pt,-2.5pt) rectangle +(2.5pt,2.5pt);

  %% insertion order
  \draw[->,dashed, color=darkblue] (0pt, 1.75 * \Y pt) node[anchor=north west, align=left]{\ \ \DARKBLUE{ordre}\\ \DARKBLUE{d'insertion}} --  (45pt, 1.75 * \Y pt);

  \draw[->,dashed, color=darkblue] (0pt, 1.75 * \Y pt) -- (-25pt, 4.75 * \Y pt);

\end{tikzpicture}

  \end{center}
  
  \vspace{0.5cm}
  
  La taille des chemins impacte les performances de l'éditeur et le trafic
  généré par l'éditeur.
  \begin{itemize}
  \item [$\rightarrow$] \textsc{allocPath} doit allouer les plus petits chemins
    possibles.
  \end{itemize}

\end{frame}


\begin{frame}{Structure de séquences}{Définition du problème}

  \begin{problem}
    Soit la séquence d'identifiants $s(I)= id_1.id_2\ldots id_I$, et
    $s(I+1) = s(I) \cup $\textsc{insert}$(p,\, \_,\, n)$ où $p,q \in s(I)$ et
    $p<_\mathcal{I}q$. Soit $|s(I)|$ la taille de la représentation binaire de la
    séquence. La fonction \textsc{insert} doit allouer des identifiants tels que :
    \begin{equation}
      |s(I+1)| - |s(I)| < \mathcal{O}(I)
    \end{equation}
  \end{problem}
  
  \vspace{0.5cm}

  \begin{itemize}
  \item [$\rightarrow$] Sinon, il faut relocaliser les identifiants qui ont été
    générés pour conserver de bonnes performances
    \begin{itemize}
    \item [$\approx$] consensus répartis qui ne passe pas à l'échelle.
    \end{itemize}
  \end{itemize}

\end{frame}


\begin{frame}{Structure de séquences}{\LSEQ}
  
  Fonction d'allocation de chemins :
  \begin{itemize}
  \item Arbre exponentiel
  \item Sous-fonctions d'allocation
  \item Fonction commune d'assignation de sous-fonctions    
  \end{itemize}
  
\end{frame}