\section{Un protocole d'échantillonnage aléatoire adaptatif}

\begin{frame}
  Les éditeurs collaboratifs nécessitent un moyen de communiquer les changements
  effectués sur le document à tous les éditeurs impliqués dans l'édition.
  
  \begin{itemize}
  \item [$\rightarrow$] Dissémination d'information
  \end{itemize}
  \vspace{0.5cm}

  Le contexte Web pousse à la centralisation. Cela pose des problèmes de passage
  à l'échelle, notamment en nombre de collaborateurs.

  \begin{itemize}
  \item [$\rightarrow$] Diffusion épidémique de message : manière efficace et
    décentralisée de propager un message.
  \item [$\rightarrow$] Rendu possible grâce à la récente technologie WebRTC :
    un navigateur peut devenir à la fois client et serveur.
    \begin{itemize}
    \item Les nœuds n'ont pas d'adresses ni de routes, les connexions sont plus
      coûteuses et sujettes aux defaillances que sur réseau IP
    \item Les navigateurs fonctionnent sur des outils aux capacités hétérogènes
      et parfois limités.
    \item Web expose aux pics soudains de popularité
    \end{itemize}
  \end{itemize}
  

\end{frame}

